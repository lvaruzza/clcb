\documentclass{book}
\usepackage{listings}

\begin{document}
\chapter{Introduction}



\section{Motivation for CLCB}

Lisp is fun. And it was felt that for a particular problem at hand, 
there was little gained by using Bio* libraries for Java, Python or Perl.
Hence, the decision was made to implement in Lisp and from day one kept in
mind to offer the fun to others, too.

One can program Lisp as if it was Perl, which is referred to as
''declarative''. What Lisp once was special for is the {\em fun}ctional
programming. Everyone has experienced what functional programming is alike
who has specified the first argument to the {\em sort} routine in Perl,
which is the expression deciding if the objects \$a or \$b is larger, with
a non-trivial expression that possibly called another function.  In Lisp,
such functions that do not need a name are called $\lambda$ expressions.
One take such, optionally gives them a name to make them real functions,
and applies them to lists of objects. To the biologically minded in us,
such functions are much like an enzyme that is acting on some sequence of
nucleotides. Other languages have constructs that allow for a functional
programming to some degree. But with Lisp it is the other way around.

But exactly {\em why} is it so much fun? There are probably many answers
to this questions.  For one, the language is very clean, even for concepts
that came into fashion twenty years after Lisp's conception. Take
for instance the overloading of methods when introducing a subclass.
When interating over a set of zoo animals, there is no need for the
typical switch statement to learn about the object's real type and then a
cast to call the real method. If in C or Java you would do something alike

\lstset{language=C}
\begin{lstlisting}
	for(animal *a= alist->first(); a->next();) {
		if ("dog"==typeof(a)) {
			dog *d = (dog *)a;
			d->speak();
		}
		else if ("cat"==typeof(a)) {
		...
		}
	}
\end{lstlisting}

to hear the miows and barks.  In Lisp, you'd get the sound directly for
any animal, also for the ones not originally anticipated in that zoo loop.
For bioinformatics, with its zoo of sequence types and their subtypes,
the genes and what was formerly known as "junk", such cleanliness is essential.

There are other reasons....

\section{How to learn Lisp}

The Net is full of tutorials that introduce to the language
Lisp. We found the instructions at the GMU very good to follow
(http://www.cs.gmu.edu/~sean/lisp/LispTutorial.html). A much recommended
text book is

	Peter Seibel\\
	Practical Common Lisp\\
	ISBN 1-59059-239-5\\
	Apress

A more biologically inclined reader of this document may decide
to leave the syntax of the language aside while first passing through
this document and instead concentrate more on the understanding
of the classes that represent biological entities in CLCB.

\section{Instant access to CLCB without reading this documentation}

CLCB is developed with SBCL but should be compatible with other
implementations with Lisp. This section gives you a head start into the
code. Frankly, since the real documentation is only worked on.
Unless you have a colleague who is already proficient in CLCB, 
there is no alternative to jumping right in.

\subsection{Setting up the interpreter}

In your UNIX shell change the directory to where the README 
of CLCB is. The UNIX command ''pwd'' informs about the current
working directory. Then start the Lisp interpreter. In the lines below
substitute ''path/to/CLCB'' with that the command ''pwd'' just sent to
the screen.

\lstset{language=lisp}
\begin{lstlisting}
(pushnew <path/to/CLCB> asdf:*central-registry*)
(pushnew <path/to/CLCB/ensembl> asdf:*central-registry*)
(asdf:oos 'asdf:load-op :clcb) or
(asdf:oos 'asdf:load-op :clcb-ensembl)
(in-package :ensembl)
\end{lstlisting}

\subsection{First hands on experience}
The code below retrieves a gene from Ensembl, retrieves its transcripts and finally 
that first's protein protein product.

\begin{lstlisting}
ENSEMBL> (fetch-by-stable-id ``ENSG00000000005'')
=> #<GENE {C974EC1}>

(defparameter *gene* (fetch-by-stable-id ``ENSG00000000005''))

ENSEMBL> (transcripts *gene*)
=> (#<TRANSCRIPT {CAB4751}>)

ENSEMBL> (translation (car *))
=> #<TRANSLATION {CAF4C49}>
\end{lstlisting}

Retrieval of exons that contribute to the first transcript.
\begin{lstlisting}
ENSEMBL> (exons (first (transcripts *gene*)))
=> 
(#<EXON {CB4F9E9}> #<EXON {CB5C809}> #<EXON {CB69539}> #<EXON {CB76249}>
 #<EXON {CB831E9}> #<EXON {CB8FF09}> #<EXON {CB9CC09}> #<EXON {CBA98E9}>
 #<EXON {CBB6619}> #<EXON {CBC3329}> #<EXON {CBD0011}> #<EXON {CBDCD09}>
 #<EXON {CBE9A09}> #<EXON {CBF6729}> #<EXON {CC03421}> #<EXON {CC10149}>
 #<EXON {CC1CE61}> #<EXON {CC2BB61}> #<EXON {CC3A8A9}> #<EXON {CC47AD1}>
 #<EXON {CC56809}>)
\end{lstlisting}

Retrieval of features of the first transcript's protein product.

\begin{lstlisting}
ENSEMBL> (protein-features (translation (car (transcripts *gene*))))
(#<PROTEIN-FEATURE {CD15729}> #<PROTEIN-FEATURE {CD178C9}>
 #<PROTEIN-FEATURE {CD18591}> #<PROTEIN-FEATURE {CD19269}>
 #<PROTEIN-FEATURE {CD19F31}> #<PROTEIN-FEATURE {CD1ABF9}>
 #<PROTEIN-FEATURE {CD1B8C9}> #<PROTEIN-FEATURE {CD1C591}>)
ENSEMBL> (mapcar #'protein-feature-type
                 (protein-features (translation (car (transcripts *gene*)))))
("Pfam" "Superfamily" "Superfamily" "Smart" "low_complexity" "low_complexity"
 "low_complexity" "Prosite_profiles")
\end{lstlisting}

Genomic coordinates of the protein features. But where is the chromosome
and where the organism? Well, normally you know about these already and they
are invariant whenever one is inspecting a particular protein that is not
a product of a translocation .... hm .... that said ....

\begin{lstlisting}
(mapcar #'dna-sequence-interval
                 (protein-features (translation (car (transcripts *gene*)))))
(#<DNA-SEQUENCE {D1DF909}>
 #<MULTI-INTERVAL (#<DNA-SEQUENCE {D1FD389}> #<DNA-SEQUENCE {D1FE011}>)>
 #<MULTI-INTERVAL (#<DNA-SEQUENCE {D21BEF1}> #<DNA-SEQUENCE {D21CB71}> #<DNA-SEQUENCE {D21D7F1}>)>
 #<DNA-SEQUENCE {D2D2A01}> #<DNA-SEQUENCE {D2EA241}> #<DNA-SEQUENCE {D3019E1}>
 #<MULTI-INTERVAL (#<DNA-SEQUENCE {D320179}> #<DNA-SEQUENCE {D320DF9}>)>
 #<DNA-SEQUENCE {D342429}>)
ENSEMBL> (mapcar #'(lambda (x) (genomic-coordinates (dna-sequence-interval x)))
                 (protein-features (translation (car (transcripts *gene*)))))
((204639 . 204885) (204630 . 205434) (250229 . 255355) (204639 . 204918)
 (242551 . 242602) (216822 . 216855) (204921 . 205473) (204639 . 204888))
ENSEMBL> (chromosome *gene*)
"18"

\end{lstlisting}


\end{document}
