\documentclass{bioinfo}
\copyrightyear{2008}
\pubyear{2008}

\usepackage{tikz}
\usetikzlibrary{shapes}
\usepackage{fancyvrb}
\usepackage{subfigure}
\usepackage[british]{babel}

%\date{\today}

\begin{document}

\firstpage{1}

\title[CLCB: Lisp for Computational Biology]{CLCB: Coherent Lisp access to resources in Computational Biology}
\author[A. Krewinkel and S. M{\"{o}}ller]{Albert Krewinkel\footnote{To whom correspondence should
        be addressed}\ \   and Steffen M{\"{o}}ller}
\address{University of L{\"{u}}beck, Institute for Neuro- and Bioinformatics, 23538 L\"ubeck, Germany}

\maketitle

\begin{abstract}
\section{Summary:}
This work introduces a Lisp package for accessing information in the
genome database EnsEMBL and thereby integrating data of SNPs, genes,
transcripts and proteins. 
%
For the programmer this work transforms the Ensembl core databases
into structures that are intuitively accessed much like the BioMart
warehouse. However, these links remain dynamic and may be extended
smoothly across databases and arbitrary external data sources.
%
CLCB further provides an additional abstract level to work with those
biological objects that have no database link. Sequence features are
modelled as intervals and set operations have been defined on these
allowing for sophisticated filters across all biological domains.

\section{Availability:} \href{http://clcb.sf.net}{http://clcb.sf.net}
\section{Contact:} \href{krewink@inb.uni-luebeck.de}{krewink@inb.uni-luebeck.de}

\end{abstract}

\section{Introduction}

Computational biology experiences a steady gain not only in the sheer
volume of data that needs to be processed. The data that is stored
becomes more and more interconnected as well. For instance, to investigate
the impact of genomic variations, one will address sources of sequence
similarity for intergenomics, retrieve protein features possibly affected
and retrieve expression data for the gene itself and its neighbours in
molecular pathway  databases.

The volume of data is coped with the old technology of relational
databases and since the IT still develops faster than that data rate, this
is save. The models for interactions of the data, however, need to be
transferred between the human and the machine. The community approaches
behind BioJava \citep{biojava}, BioPerl \citep{bioperl:2002} or BioRuby
\citep{bioruby} have provided an enormous wealth of
tricks for dealing with data in computational biology. These however are not
meant to express complex data structures. The agent community has not yet
presented a solution either \citep{agents:2007}.

The motivation behind the here presented work lies in the hunch that
by strengthening the functional aspects of programming languages, this
situation can be helped.  A technology older than relational databases,
the programming language Lisp, was employed to model interactions
between biological entities in the EnsEMBL \citep{BiAnCaChClCoCoCuCuCuDoDuFeFlGrHaHeHoHu06}
core and SNP databases. To
the end use this functions much like prepared tables of BioMart but
without constraints on the semantic links between the data entries.

\section{Approach}

The here presented work has two layers. One for the abstract representation of biological enties,
the other for the link of these to the Ensembl database. Every table
in Ensembl is represented as a Lisp class. When one would express a
join statement in SQL, the Lisp classes provide a member function to
dynamically retrieve thus linked objects. This principle was applied to
the core databases, compara and variation.


\section{Methods}
\begin{methods}
The development was performed with Debian Linux for SBCL and Common Lisp. The Ensembl
compara and core MySQL databases have been installed locally, which is not required
for the use of this library. The databases are accessed with the CLSQL
package\footnote{See \href{http://clsql.b9.com/}{http://clsql.b9.com/}}.
\end{methods}

\section{Discussion}

The common myths about Lisp have all long been debunked\footnote{See
\href{http://www.lispworks.com/products/myths\_and\_legends.html}{http://www.lispworks.com/products/myths\_and\_legends.html}}.
Lisp integrates principles from functional, iterative and object-oriented
programming and its macro system provides it with an unparalleled
flexibility. When applying functional programming techniques, the computational biologist
follows the data through the code and operations may be passed as
entities between functions. This allows to
think about the function of, e.g. a nuclease, in complete analogy to
the enzyme on a molecular level: both can be added to a set of target
sequences.

The CLCB library is moderately complete for its application on the EnsEMBL
database. The success that the statistics envirionment R \citep{Rbioinformatics}
has in the field of bioinformatics, which shared many concepts of a
functional language, is indicative of the potential that Lisp has for
computational biology.


\bibliographystyle{natbib}
\begin{methods}
\bibliography{clcb}
\end{methods}

\end{document}
